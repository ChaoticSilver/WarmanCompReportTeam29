%Metadata
\documentclass[12pt]{report}
\usepackage[english]{babel}
\usepackage[utf8x]{inputenc}
\usepackage{booktabs} %Table
\usepackage{multirow} %Rows
\usepackage{graphicx} %Insert Pictures
\usepackage{enumitem} %ToC
\usepackage{titlesec} 
\usepackage{geometry} %Margins

%=========== DISABLE THIS WHEN EXPORT... Remove the %
\setkeys{Gin}{draft=False}


\titleformat{\chapter}{\normalfont\huge}{\thechapter.}{20pt}{\huge\it}
\titlespacing*{\chapter}{0pt}{50pt}{40pt}
\titlespacing*{name=\chapter}{0pt}{-30pt}{10pt}

\renewcommand{\thetable}{\Alph{table}}
\title{Warman Competiton Report}
\linespread{1.5}
\begin{document}


%----------------------------------------------------------------------------------------
%	TITLE PAGE
%----------------------------------------------------------------------------------------
	
\begin{titlepage}
\pagenumbering{roman} %Use roman letters for number at the start
\newcommand{\HRule}{\rule{\linewidth}{0.5mm}}

\center % Center everything on the page

%=================HEADING =======================================
\includegraphics{UoA_logo.jpg}\\
\textsc{\large School of Mechanical Engineering}\\[1cm]
\textsc{\Huge Design Practice}\\
\textsc{\large Mech Eng 2100}\\

%===============TITLE =======================

\HRule \\[0.4cm]
{ \Huge \bfseries Warman Competition}\\[0.2cm]
{ \Huge \bfseries Final Report}\\[0.1cm]
\HRule \\[-0.2cm]

%================NAMES ==========================

\huge TEAM 29 \\ [-0.2cm]
\Large The Professional Mechanical Teletubbies \\[1 cm]

\begin{minipage}{0.5\textwidth}
\begin{flushleft} \small
  \textsc{Xing Yong TAN}
   \linebreak
 \textsc{a1732497}
 \linebreak
  \linebreak
  \textsc{Shanmughanathan LAKSHMANAN}
   \linebreak 
 \textsc{a1741366}
  \linebreak
  \linebreak
  \textsc{Hamza Saquib Zain KHAN}
  \linebreak
  \textsc{a1741366}
 
\end{flushleft}
\end{minipage}
~
\begin{minipage}{0.4\textwidth}
\begin{flushright} \small
  \textsc{Peng Hoe HOR}
   \linebreak
 \textsc{a1732412}
 \linebreak
  \linebreak
  \textsc{Zheng Bing LIM}
   \linebreak
 \textsc{a1732649}
 \linebreak
 \linebreak
  \linebreak
\end{flushright}
\end{minipage}\\[0.5cm]
\end{titlepage}

%----------------------------------------------------------------------------------------
%	EXECUTIVE SUMMARY
%----------------------------------------------------------------------------------------
\section*{Executive Summary}
TO BE INSERTED
\setcounter{page}{2} %Set page number to 2 as title page dont count 
\pagebreak

%----------------------------------------------------------------------------------------
%	Table of Content
%----------------------------------------------------------------------------------------
\section*{Table of Contents}
\begin{tabular}{ p{0.9\textwidth} p{0.1\textwidth} }

%TOC Details
\begin{enumerate}
    \item []Executive Summary 
    \item []Acknowledgements 
    \item Introduction 
	\begin{enumerate}
	\item [1.1] Problem specification
	\item [1.2]	Literature Review    
	\end{enumerate}
\item Design Alternatives 
	\begin{enumerate}
	\item [2.1]	Design A 
	\item [2.2]	Design B 
	\item [2.3]	Design C 
	\end{enumerate}
\item Final Design 
	\begin{enumerate}
	\item [3.1] Design Concept
    \item [3.2] Testing
    \item [3.3] Modifications
	\end{enumerate}
\item Conclusion
\item Reference List
\end{enumerate}

&
%Page numbers. Custom linewidth 
\begin{flushright}
i \\[0.2\linewidth]
ii \\[0.2\linewidth]
\end{flushright}
\end{tabular}
\pagebreak

\begin{tabular}{ p{0.9\textwidth} p{0.1\textwidth} }
Appendix
\begin{enumerate}[label=\Alph*.]
\item Project Planning and Management
\item Problem Solving and Team Reflection
\item Comparison Matrix 
\item Project Management Tools 
\item Supplementary Drawings 
\item Glossary of Terms 
\end{enumerate}
&
\begin{flushright}
\vspace{-0.3cm}
\item []  19 \\[0.25\linewidth]
\item []  20 \\[0.3\linewidth]
\end{flushright}

\tabularnewline
\end{tabular}
\pagebreak

%----------------------------------------------------------------------------------------
%	ACKNOWLEDGMENTS
%----------------------------------------------------------------------------------------
\section*{Acknowledgements}
We value the wonderful opportunity to take on this project which gave us experience in the field of robotics. 

Thanks lecturer Dr. Antoni Blazewicz
Thank tutors Nathaniel Shearer, Saxon A. Nelson-Milton, Harrison Schuster, Thomas Jacquier, Joshua Leyshon

Thank workshop technicians ????? ???? ?????  Please go find out whoever that used workshop services
\pagebreak

%----------------------------------------------------------------------------------------
%Introduction
%----------------------------------------------------------------------------------------
\pagenumbering{arabic} %Change to numerals
\chapter{Introduction}

The planet is facing a drought due to the release of methane gas. To overcome this problem, a power-pack is required to be transferred to a bunker to activate an aquifer so that the drought can be stopped. This chapter aims to introduce the assigned engineering problems, the aims and objectives in creating an autonomous vehicle to transport the power-pack to the bunker.

\section{Problem Specification} 
One of the main goals in this project is to get the payload from a pole to the bunker at the end of
the track. During the travel, the payload cannot be dropped and has to maintain the correct
orientation up until it is dropped into the bunker.
The problem can be divided into 3 parts, the first of which is for the robot to be able to
manoeuvre and travel to the correct position and orientation on the competition track. Based on
the competition rules each robot can be up to 6kg. However, the weight of the robot needs to be
optimized so that it can move quickly to maximize time score.
The second part of the problem is the collecting mechanism of the payload. There are 3
options/locations that the power packs are located all of which are of different heights. Each
height provides a different point system with the tallest pole providing the most point, the
medium pole providing 2/3 of the maximum points and the shortest pole providing the least with
1/3 of the given points. For the first two poles, the design requires the robot to extend to a certain
height such that it can collect payloads on top of the poles taller than the robot to maximize
points per Warman competition rules. For the shortest pole it does not require for the design of
the robot to extend as it is almost at ground level. To collect the power pack for the pole the
design of the robot is required to be able to collect the power pack at base height. Moreover, an
important point to note is that the collecting mechanism needs to make sure the payload is secure
so that the golf balls do not fall off to avoid a deduction in points. The constraint for this
mechanism is that the robot cannot be too large as it needs to fit through a 400x400 mm garage
gate.

The final part of the problem is the release mechanism. Firstly, it must be able to secure the
power pack in place to ensure that the arrow on the power pack is pointed in the correct
direction. If the power pack is in the wrong direction when being placed into the bunker less
points will be achieved per Warman competition rules. The robot will need to move towards the
bunker after collecting the power pack and orientate itself so that the power can be placed with
the arrow in the right direction. The bunker has steps of certain height before it. This requires for
the design of the releasing mechanism of the robot to be able to go over the steps in other to
place the power pack into the bunker. The height of the step and the distance from the outside of
the terrace to the bunker must be considered so that the design can be placed successfully into
the bunker. The method of placing the power pack itself needs to be considered as the bunker has
a size of 150 x 150 mm, the power packs needs to be placed within these constraints to maximise
the points per Warman competitor rules. Placing it anywhere else will provide lesser points as
compared to placing it into the bunker.

\section{Literature review} 
The wheels play an important part in the robot as it needs to move the robot and withstand a specific amount of torque. 


- Wheels DC Motor 


- Drivers


-Materials


-Torque on the Wheels


%Linespace is needed to prevent LaTeX from auto justifying everything.. so we cant use linebreaks
As a result of researching, it shows that each component plays different roles but they all coordinates with one and another to allow movement.
The constraints and specifications also have to be considered while manufacturing the wheels.

%----------------------------------------------------------------------------------------
%Alternative designs
%----------------------------------------------------------------------------------------
\chapter{Alternative designs}
\section{Design A - 8 wheels design}
After brainstorming through the initial stages of the robot building design procedure, the team had initially planned on making the robot with 8 wheels to achieve the aims of the competition in a timely manner {2 sets of 4 wheels in which one of the sets operate forwards/backwards while the motion of the other set of wheels operate left/right}. The robot mainly consists of a scissor lift, an extendable incline, a box capable of storing the power-pack and a base mounted on 8 wheels with 4 axles (each axle supporting 2 wheels) to hold the structure and electronics comprised in programming the robot. 
However, the 8–wheels design has some advantages and disadvantages which attribute to the robot’s core functionality.  The advantages includes high structural integrity to support the infrastructure on top of the base of the robot, total time accumulated to finish the course decreases, less programming is required to move the robot which was one of the main reasons that this design was proposed initially; the robot needs to successfully move the power-pack with the balls in place into the bunker along with the arrow on the power-pack pointing the same direction as it was picked up into the bunker for maximum points. For this, the team thought rotating the robot in a desired way will consume a lot of time as well as money to buy several motors and motor shields. Thus, the idea to build an 8–wheels design where no rotations is needed, and the robot primarily moves with 2 degrees of freedom (horizontal and vertical directions [x-y axis]) which does the job efficiently and effectively such that the power pack’s direction does not change during any part of the process. The power pack is collected via the extended arms thus probing the robot to change transmission allowing it to move towards the bunker. Once close to the bunker the original transmission wheels will move to close the distance and the motor on the of the box will activate rolling the incline plane down releasing an extended length of surface allowing the power pack to slide down precisely into the bunker with the arrow in the right direction. 
After more planning and advice from experienced people in the field of robotics, it was discovered that to do this, a lot of money would be required in order to build a transmission system such that the robot switches between the 2 sets of wheels during the course of its journey which also requires a lot of physical work, hence, the team decided that this design shall be disregarded due to the nature of difficulty in building this system along with its incorporated costs.

%%Insert Pictures

%===================Design B=============================================
\section{Design B - 4 Wheel design}
This robot is a system with a free moving arm which acts as both the lifting and the grabbing
mechanism. The robot mainly consists of an extended arm comprised of two sections so that it is
long enough to maneuver around the pole and collect the power pack and to also place it into the
bunker, a fork shaped platform to latch onto the payload whilst keeping the power pack in place
and a base mounted on 4 wheels with 2 stepper motors and two omni-directional wheels to hold
the structure and electronics comprised in programming the robot.
The 4-wheel arm system is more mobile and if executed will enable us more mobility to
approach the power pack. It also enables easier placement of power pack into the bunker as it is
able to rotate over the pole and position itself into bunker with precision. It is also able to rotate
thus giving it more options to go in several directions. The 4-wheel arm system is also light-
weight as it requires less raw material. However, this system will require more motors and motor
shields to operate the individual parts thus increasing the cost of the robot. The increased
complexity in programming is also a disadvantage as several rotations, movements and
maneuvers (with the arm) will need to be executed with high precision especially without the aid
of sensors with the current design. The electronics will also be an issue as connecting each parts
without them interfering with each other will need to be considered so that maximum efficiency
will be achieved.
The hands are extended and will rotate over the pole and sweeping the power pack into the
shaped platform, and will rotate over to the bunker side so that it can be placed with the arrow in
its initial position. Overall, the 4-wheel system provides good structural integrity, high mobility

and good maneuverability and hence is a more viable design which can be implemented in order
to achieve the aims of the competition.
%%Insert Pictures

%===================Design C=============================================
\section{Design C - 2 Wheels Design}
The 2-wheel robot system is a combination of the 8-wheels robot and the 4-wheel robot. Like the 8-wheel robot it has an extended claw mechanism to pick up the power pack and store it in a safe and secure space while waiting to be deposited into the bunker. Like the 4-wheels design however it has great maneuverability as it uses two stepper motors allowing it to rotate and move in any direction required.

 However, the 2 –wheel robot system has some advantages and disadvantages which attribute to the robot’s core functionality.  The disadvantages include less structural integrity to support the infrastructure on top of the base of the robot as it is not balanced and will tip over on one side. This will cause uncertainty in being able to keep the golf balls in the power pack as it is unstable. The total time accumulated to finish the course increases as well due to increased movement, rotations and also requires more torque to move as there will be drag from one side of the base touching the ground. It will also require some complex programming due to the movements and rotations that is required to complete the journey. Advantages of this system include less weight and less cost as only 2 stepper motors will be required and less maintenance issues as opposed to the other design proposals. It will also be easier to manufacture as it is a simple design that is able to be built using everyday materials.
The 2-wheels robot will collect the payload allowing it to slide at an angle and be blocked by a wall attached to a wire. It will rotate and travel towards the bunker. Once it reaches the bunker the robot will rotate to orientate itself so that the wall is facing the bunker and the motor will turn, lowering the wall allowing the power pack to slide into the payload.
Hence, considering the pros and cons of the design, it was disregarded due to its low reliability in stability as keeping the payload in place is crucial and worries of damage to the stepper motors while dragging the base through the floor is a point to consider as well.

%%Insert picture
%%Insert table

%----------------------------------------------------------------------------------------
%Alternative designs
%----------------------------------------------------------------------------------------


\end{document}
